%%% Definierungen
%%% Nicht vergessen: In Präambel und Main Einstellungen aktivieren um Glossar und / oder Abkürzungen zu verwenden!

%\newacronym{abk:mp3}{MP3}{MPEG Audio Layer III}


%\newglossaryentry{backbone}{
%name=Backbone,
%description={Zentrales Computernetzwerk des Internetanbieters über welches der Datenverkehr geleitet wird},
%user1={Backbones}}


%% Definierung von Abkürzungen
% \newacronym{Label}{Abkürzung}{Voller Name}

%% Definierung von Abkürzungen mit Glossareintrag
% \newacronym{Label für Abkürzung}{Abkürzung}{Bedeutung\protect\glsadd{Label für Glossar-Eintrag}}

%% Definierung von Glossar-Einträgen. Plural Name optional. Wenn spezial Befehl in Präambel gesetzt, kann die Beschreibung mit einem Punkt geschlossen werden. Ansonsten keinen setzen.
% \newglossaryentry{Label}{
% name=Name Singular,
% description={Beschreibung},
% user1={Name Plural}}


%%% Nutzung im Text

%% Glossar-Einträge. Immer verwenden da Seitenzahlen der Verwendung im Glossar angegeben werden. Wort wird durch Befehl gesetzt.
% Name wird in Singular im Text gesetzt: \gls{Label}
% Name wird in Plurar im Text gesetzt: \glsuseri{Label}

%% Abkürzungen. Erstmalige Verwendung wird das Wort ausgeschrieben und in Klammer die Abkürzung erwähnt. Im Weiteren Verlauf nur die Abkürzung. Wort wird durch Befehl gesetzt. Seitenzahlen erscheinen im Verzeichnis.
% \gls{Label}

%% Abkürzungen mit Glossareintrag. Nicht Glossar-Label benutzen. Wie Abkürzung nur wird gleichzeitig ein Glossar-Eintrag gepflegt.
% \gls{Label}