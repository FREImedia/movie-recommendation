%% Ermöglicht die Kompilierung durch Sublime Text von jeder *.tex Datei aus.
%!TEX root = main.tex

%%% PDF Erstellung mit Glossar
%%% 1 . Kompilieren durch Sublime Tex
%%% 2. Folgende Befehle im Terminal eingeben
%%%	pdflatex main
%%%	makeindex -s main.ist -t main.alg -o main.acr main.acn
%%%	makeindex -s main.ist -t main.glg -o main.gls main.glo
%%%	makeindex -s main.ist -t main.slg -o main.syi main.syg
%%%	pdflatex main

%%% Für schöneren Index wird die Datei index.ist benötigt. Im Terminmal "makeindex -g -s index.ist main" ausführen danach normal kompilieren (davor vielleicht auch schon mal).

%%% Bei Beschreibungen für [] und {}: Durch Komma getrennte Beschreibungen zeigen auf die in [] optionalen Parameter. Nach Semikolon Beschreibungen für den Parameter in {}. Ohne Kommatas = allgemeine Beschreibung. Nur kommatas = Beschreibung der Optionen in [] oder {}.

%% Schriftgröße 12, Format A4, Aufnahme des Abbildungsverzeichnis ins Inhaltsverzeichnis, Aufnahme des Literaturverzeichnis ins Inhaltsverzeichnis, Kapitel und Inhaltsverzeichnis u.a. beginnen auf der rechten Seite, Richtige Seitenzahl des Index im Inhaltsverzeichnis; Für längere Berichte u.a. Titel auf eigender Seite.
%\documentclass[12pt,a4paper,liststotoc,bibtotoc,openright]{report}
\documentclass[12pt,a4paper,liststotoc,bibtotoc,openright]{report}

\usepackage{titlesec} % Textüberschriften anpassen
% \titleformat{⟨Überschriftenklasse⟩}[Absatzformatierung⟩]{⟨Textformatierung⟩} {⟨Nummerierung⟩}{⟨Abstand zwischen Nummerierung und Überschriftentext⟩}{⟨Code vor der Überschrift⟩}[⟨Code nach der Überschrift⟩]

%\titleformat{\chapter}[hang]{\large\bfseries}{\thechapter\quad}{0pt}{}
\titleformat{\chapter}[display]   
{\normalfont\huge\setstretch{0.35}}{\chaptertitlename\ \thechapter}{20pt}{\bfseries\Huge}   
%\titleformat{\section}[hang]{\large\bfseries}{\thesection\quad}{0pt}{}
%\titleformat{\subsection}[hang]{\large\bfseries}{\thesubsection\quad}{0pt}{}
%\titleformat{\subsubsection}[hang]{\large\bfseries}{\thesubsubsection\quad}{0pt}{}
%\titleformat{\paragraph}[hang]{\large\bfseries}{\theparagraph\quad}{0pt}{}

% \titlespacing{⟨Überschriftenklasse⟩}{⟨Linker Einzug⟩}{⟨Platz oberhalb⟩}{⟨Platz unterhalb⟩}[⟨rechter Einzug⟩]

%\titlespacing{\chapter}{0pt}{-6em}{6pt}
\titlespacing*{\chapter}{0pt}{-60pt}{30pt}
%\titlespacing{\section}{0pt}{6pt}{6pt}
%\titlespacing{\subsection}{0pt}{6pt}{6pt}
%\titlespacing{\subsubsection}{0pt}{6pt}{6pt}
%\titlespacing{\paragraph}{0pt}{6pt}{6pt}


% Deaktiviere leere Seite nach jedem Chapter
\let\cleardoublepage\clearpage


%% Korrekte Kodierung u.a. von Umlauten
\usepackage[T1]{fontenc}

%% Ermöglicht u.a. das direkte tippen von Umlauten; Plattformunabhängige Kodierung.
\usepackage[utf8]{inputenc}

%% Englische Rechtschreibung, Neue deutsche Rechtschreibung; u.a. übersetzt automatisch generierte Dokumentelemente in die deutsche Sprache
\usepackage[english, ngerman]{babel}

%% Schriftart Helvetica
\usepackage{helvet}

%% Seitenrand links=innen=3cm, rechts=außen=2cm 
%\usepackage[left=3cm,right=2cm]{geometry}
\usepackage[]{geometry}


%% Zeilenabstand 1,5
\usepackage[onehalfspacing]{setspace}

% Schusterjungen und Hurenkinder vermeiden
\clubpenalty = 10000
\widowpenalty = 10000 
\displaywidowpenalty = 10000

% Überstehende Textteile verhindern
\sloppy

% Kopf- und Fußzeilen
\usepackage{scrpage2}
\pagestyle{scrheadings}
\clearscrheadfoot
\automark[section]{chapter}


\renewcommand{\headfont}{\normalfont} % Schriftform der Kopfzeile
\ihead{\headmark}
\chead{}
\setlength{\headheight}{21mm} % Höhe der Kopfzeile
%\setheadwidth[0pt]{textwithmarginpar}  % Kopfzeile über den Text hinaus verbreitern
\setheadsepline[text]{0.4pt} % Trennlinie unter Kopfzeile

\ifoot{}
\cfoot{}
\ofoot[\pagemark]{\pagemark}


% Ermöglicht Tables
\usepackage{booktabs}

%% Ermöglicht Stichwortverzeichnisses (Index)
\usepackage{makeidx}

%% Stichwortverzeichnis (Index) erstellen
\makeindex

%% Ermöglicht die Einbindung von Grafiken
\usepackage{graphicx}

%% Ermöglicht Farben
\usepackage{color}
\definecolor{light-gray}{gray}{0.925}

%% Ermöglicht Listings
\usepackage{listings} 

%% Zeilennummer links, Zeilennummer-Größe tiny und Schriftart RM, Größe des Listing-Texts tiny und Schriftart RM, Titel unterhalb des Listings. Weiteres kann Einbindung des Listings definiert werden.
\lstset{basicstyle=\ttfamily, numbers=left, breaklines=true, captionpos=b, language=python, frame=single, backgroundcolor=\color{light-gray}, showstringspaces=false} 

\usepackage{listings}
\renewcommand{\lstlistingname}{Codebeispiel} % Listing -> Codebeispiel
\usepackage[labelfont=bf]{caption} % Bezeichnung und Nummer fett

%% Vermeidet die Umrandung von Links im PDF; Ermöglicht u.a. Umbrüche bei URLs und Erstellung Hyperlinks
\usepackage[hidelinks]{hyperref}

%% Erstellung eines Abkürzungsverzeichnis, Eintrag ins Inhaltsverzeichnis; Erstellung eines Glossar
\usepackage[acronym,toc,nonumberlist]{glossaries}

%% Nummerische Zitierung; Naturwissenschaftliches zitieren
\usepackage[numbers]{natbib}

%%% Zweites Literaturverzeichnis z. B. für Bilder (Bildqellenverzeichnis)
%% Mehrere Literaturverzeichnisse
% \usepackage{multibib}
%% Definierung eines Flag, Name des Verzeichnis
% \newcites{B}{Bildquellenverzeichnis}

%% Ermöglicht einfache Darstellung von römischen Zahlen. Nutzung z.B. \RM{12} Wird ein . nach der Römischen Zahl benötigt muss der Teil #1{} so aussehen: #1{.}
\newcommand{\RM}[1]{\MakeUppercase{\romannumeral #1{}}}

%% Verkürzt den Abstand bei Benutzung von \chapter zum oberen Seitenrand
\def\chapterheadstartvskip{\vspace*{-\normalbaselineskip}\vspace*{-\topskip}}

%% Ein eigenes Symbolverzeichnis erstellen
% \newglossary[slg]{symbolslist}{syi}{syg}{Symbolverzeichnis}
 
%% Den Punkt am Ende jeder Beschreibung deaktivieren (Damit nicht zwei platziert werden wenn man bei der Beschreibung einen setzt)
\renewcommand*{\glspostdescription}{}

%% Glossar-Befehle aktivieren
\glsaddall
\makeglossaries

%% Bindet die Verwendeteten Glossar-Labels ein
%%% Definierungen
%%% Nicht vergessen: In Präambel und Main Einstellungen aktivieren um Glossar und / oder Abkürzungen zu verwenden!

%\newacronym{abk:mp3}{MP3}{MPEG Audio Layer III}


%\newglossaryentry{backbone}{
%name=Backbone,
%description={Zentrales Computernetzwerk des Internetanbieters über welches der Datenverkehr geleitet wird},
%user1={Backbones}}


%% Definierung von Abkürzungen
% \newacronym{Label}{Abkürzung}{Voller Name}

%% Definierung von Abkürzungen mit Glossareintrag
% \newacronym{Label für Abkürzung}{Abkürzung}{Bedeutung\protect\glsadd{Label für Glossar-Eintrag}}

%% Definierung von Glossar-Einträgen. Plural Name optional. Wenn spezial Befehl in Präambel gesetzt, kann die Beschreibung mit einem Punkt geschlossen werden. Ansonsten keinen setzen.
% \newglossaryentry{Label}{
% name=Name Singular,
% description={Beschreibung},
% user1={Name Plural}}


%%% Nutzung im Text

%% Glossar-Einträge. Immer verwenden da Seitenzahlen der Verwendung im Glossar angegeben werden. Wort wird durch Befehl gesetzt.
% Name wird in Singular im Text gesetzt: \gls{Label}
% Name wird in Plurar im Text gesetzt: \glsuseri{Label}

%% Abkürzungen. Erstmalige Verwendung wird das Wort ausgeschrieben und in Klammer die Abkürzung erwähnt. Im Weiteren Verlauf nur die Abkürzung. Wort wird durch Befehl gesetzt. Seitenzahlen erscheinen im Verzeichnis.
% \gls{Label}

%% Abkürzungen mit Glossareintrag. Nicht Glossar-Label benutzen. Wie Abkürzung nur wird gleichzeitig ein Glossar-Eintrag gepflegt.
% \gls{Label}